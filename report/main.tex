%%%%%%%% Sample LaTeX input for Complex Systems %%%%%%%%%%% 
% Revision 4, Jun 27, 2018
%
% This is a LaTeX input file  
% Text following % on a particular line is treated as a comment, and 
% ignored by LaTeX.  
% You do not need to type any text that follows a % 
% 
\documentclass[10pt]{article}

\usepackage[a4paper, margin=1.5in]{geometry}
\usepackage{float}

\usepackage{epsf,hyperref}
\usepackage{amssymb,ComplexSystems}
\usepackage{lmodern}  % Improved font for better scalability

% Font Encoding and Input
\usepackage[T1]{fontenc}  % Updated to use T1 encoding for better font rendering
\usepackage[utf8]{inputenc}  % Use utf8 instead of utf8x for better compatibility

% Language
\usepackage[english]{babel}

\usepackage{amsmath}

\usepackage{listings}
\usepackage{xcolor}

\definecolor{codegreen}{rgb}{0,0.6,0}
\definecolor{codegray}{rgb}{0.5,0.5,0.5}
\definecolor{codepurple}{rgb}{0.58,0,0.82}
\definecolor{backcolour}{rgb}{0.95,0.95,0.92}

% Code Listings
\usepackage{listings}
\usepackage{color}
\usepackage[table,xcdraw]{xcolor}

% Graphics and Figures
\usepackage{graphicx}
\usepackage{wrapfig}
\usepackage{subcaption}
\usepackage{caption}

% Clever References
\usepackage{cleveref}

\usepackage{calrsfs}
\DeclareMathAlphabet{\pazocal}{OMS}{zplm}{m}{n}
\newcommand{\La}{\mathcal{L}}
\newcommand{\Lb}{\pazocal{L}}

% Bibliography
\usepackage{biblatex}
\addbibresource{biblio.bib}

\lstset{
tabsize = 4, %% set tab space width
showstringspaces = false, %% prevent space marking in strings, string is defined as the text that is generally printed directly to the console
numbers = left, %% display line numbers on the left
commentstyle = \color{green}, %% set comment color
keywordstyle = \color{blue}, %% set keyword color
stringstyle = \color{red}, %% set string color
rulecolor = \color{black}, %% set frame color to avoid being affected by text color
basicstyle = \small \ttfamily , %% set listing font and size
breaklines = true, %% enable line breaking
numberstyle = \tiny,
}

% complex-systems.sty is the macro package for Complex Systems.
% It is available at
% http://www.complex-systems.com/samples/complex-systems.sty
% epsf.sty is the preferred graphics import method

\begin{document}

\title{Secure Cloud Computing:\\ {\large ORAM and Homomorphic Encryption}% 
% Use \\ to indicate line breaks in titles longer than about 
% 55 characters. 
%
}

\author{\authname{Riccardo Gennaro}\\[2pt] 
% Use \\[2pt] to end the line and add space between author name and affiliation. 
\authadd{Student ID: 3534219}\\
\authadd{Group 5}\\
\and
\authname{Péter Svelecz}\\[2pt] 
% Use \\[2pt] to end the line and add space between author name and affiliation. 
\authadd{Student ID: 3542629}\\
\authadd{Group 5}\\
}

% The following specifies the running headings 
%
% Each running heading should be less than about 50 characters long. 
% If necessary, give a shortened version of the title. 
%
% Use initials for first and second names. If all author names do not fit, truncate the 
% list and end with ``et al.''.
\markboth{Secure Cloud Computing 2024 Assignment 2} 
{ORAM and Homomorphic Encryption} 

\maketitle
% End title section

%\begin{abstract}
%    \section{Path ORAM}

\subsection{Simulation results for Path ORAM}

Following the requirements for this assignment, we tested our implementation for a number of blocks $N=2^(15)$. The test consisted of two runs with a warmup of $3*10^6$ write accesses and an actual simulation of $3*10^6$ read accesses. The first simulation was carried out with a number of blocks per bucket $Z=2$ while the second with $Z=4$.

Following, you can find the results for $Z=2$ formatted as per instructions.

\begin{verbatim}
For Z=2
    -1,3000000
    0,3000000
    1,2083373
    2,792131
    3,306820
    4,122796
    5,51130
    6,22367
    7,10150
    8,4804
    9,2381
    10,1240
    11,654
    12,359
    13,179
    14,70
    15,27
    16,6
    17,3
    18,1
\end{verbatim}

Output can be found in files \texttt{simulation1.txt} and \texttt{simulation2.txt} for $Z=2$ and $Z=4$ respectively.

Gicen the lenght of the simulation for $Z=4$, its output is not reported in this document.

\subsection{Probability of stash overflow}

For both simulations, we map the probabilty of stash overflow given a stash lenght contraint. Following the results for $Z=2$ and $Z=4$ respectively.

\begin{figure}[H]
    \centering
    \includegraphics[width=\textwidth]{02-ex1/plot_z_2.png}
    \caption{Probabilty of stash overflow for $Z=2$.}
    \label{fig:stash-overflow-for-Z=2}
\end{figure}

\begin{figure}[H]
    \centering
    \includegraphics[width=\textwidth]{02-ex1/plot_z_4.png}
    \caption{Probabilty of stash overflow for $Z=4$.}
    \label{fig:stash-overflow-for-Z=4}
\end{figure}
%\end{abstract}

%\begin{keywords}
%    Personal Health Record, PHR, Access Control, dCP-ABE
%\end{keywords}

\section{Path ORAM}

\subsection{Simulation results for Path ORAM}

Following the requirements for this assignment, we tested our implementation for a number of blocks $N=2^(15)$. The test consisted of two runs with a warmup of $3*10^6$ write accesses and an actual simulation of $3*10^6$ read accesses. The first simulation was carried out with a number of blocks per bucket $Z=2$ while the second with $Z=4$.

Following, you can find the results for $Z=2$ formatted as per instructions.

\begin{verbatim}
For Z=2
    -1,3000000
    0,3000000
    1,2083373
    2,792131
    3,306820
    4,122796
    5,51130
    6,22367
    7,10150
    8,4804
    9,2381
    10,1240
    11,654
    12,359
    13,179
    14,70
    15,27
    16,6
    17,3
    18,1
\end{verbatim}

Output can be found in files \texttt{simulation1.txt} and \texttt{simulation2.txt} for $Z=2$ and $Z=4$ respectively.

Gicen the lenght of the simulation for $Z=4$, its output is not reported in this document.

\subsection{Probability of stash overflow}

For both simulations, we map the probabilty of stash overflow given a stash lenght contraint. Following the results for $Z=2$ and $Z=4$ respectively.

\begin{figure}[H]
    \centering
    \includegraphics[width=\textwidth]{02-ex1/plot_z_2.png}
    \caption{Probabilty of stash overflow for $Z=2$.}
    \label{fig:stash-overflow-for-Z=2}
\end{figure}

\begin{figure}[H]
    \centering
    \includegraphics[width=\textwidth]{02-ex1/plot_z_4.png}
    \caption{Probabilty of stash overflow for $Z=4$.}
    \label{fig:stash-overflow-for-Z=4}
\end{figure}
\newpage
\section{Path ORAM}

\subsection{Simulation results for Path ORAM}

Following the requirements for this assignment, we tested our implementation for a number of blocks $N=2^(15)$. The test consisted of two runs with a warmup of $3*10^6$ write accesses and an actual simulation of $3*10^6$ read accesses. The first simulation was carried out with a number of blocks per bucket $Z=2$ while the second with $Z=4$.

Following, you can find the results for $Z=2$ formatted as per instructions.

\begin{verbatim}
For Z=2
    -1,3000000
    0,3000000
    1,2083373
    2,792131
    3,306820
    4,122796
    5,51130
    6,22367
    7,10150
    8,4804
    9,2381
    10,1240
    11,654
    12,359
    13,179
    14,70
    15,27
    16,6
    17,3
    18,1
\end{verbatim}

Output can be found in files \texttt{simulation1.txt} and \texttt{simulation2.txt} for $Z=2$ and $Z=4$ respectively.

Gicen the lenght of the simulation for $Z=4$, its output is not reported in this document.

\subsection{Probability of stash overflow}

For both simulations, we map the probabilty of stash overflow given a stash lenght contraint. Following the results for $Z=2$ and $Z=4$ respectively.

\begin{figure}[H]
    \centering
    \includegraphics[width=\textwidth]{02-ex1/plot_z_2.png}
    \caption{Probabilty of stash overflow for $Z=2$.}
    \label{fig:stash-overflow-for-Z=2}
\end{figure}

\begin{figure}[H]
    \centering
    \includegraphics[width=\textwidth]{02-ex1/plot_z_4.png}
    \caption{Probabilty of stash overflow for $Z=4$.}
    \label{fig:stash-overflow-for-Z=4}
\end{figure}


%%\printbibliography

\end{document}
