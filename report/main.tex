%%%%%%%% Sample LaTeX input for Complex Systems %%%%%%%%%%% 
% Revision 4, Jun 27, 2018
%
% This is a LaTeX input file  
% Text following % on a particular line is treated as a comment, and 
% ignored by LaTeX.  
% You do not need to type any text that follows a % 
% 
\documentclass[10pt]{article}

\usepackage[a4paper, margin=1.5in]{geometry}
\usepackage{float}

\usepackage{epsf,hyperref}
\usepackage{amssymb,ComplexSystems}
\usepackage{lmodern}  % Improved font for better scalability

% Font Encoding and Input
\usepackage[T1]{fontenc}  % Updated to use T1 encoding for better font rendering
\usepackage[utf8]{inputenc}  % Use utf8 instead of utf8x for better compatibility

% Language
\usepackage[english]{babel}

\usepackage{amsmath}

\usepackage{listings}
\usepackage{xcolor}

\definecolor{codegreen}{rgb}{0,0.6,0}
\definecolor{codegray}{rgb}{0.5,0.5,0.5}
\definecolor{codepurple}{rgb}{0.58,0,0.82}
\definecolor{backcolour}{rgb}{0.95,0.95,0.92}

% Code Listings
\usepackage{listings}
\usepackage{color}
\usepackage[table,xcdraw]{xcolor}

% Graphics and Figures
\usepackage{graphicx}
\usepackage{wrapfig}
\usepackage{subcaption}
\usepackage{caption}

% Clever References
\usepackage{cleveref}

\usepackage{calrsfs}
\DeclareMathAlphabet{\pazocal}{OMS}{zplm}{m}{n}
\newcommand{\La}{\mathcal{L}}
\newcommand{\Lb}{\pazocal{L}}

% Bibliography
\usepackage{biblatex}
\addbibresource{biblio.bib}

\lstset{
tabsize = 4, %% set tab space width
showstringspaces = false, %% prevent space marking in strings, string is defined as the text that is generally printed directly to the console
numbers = left, %% display line numbers on the left
commentstyle = \color{green}, %% set comment color
keywordstyle = \color{blue}, %% set keyword color
stringstyle = \color{red}, %% set string color
rulecolor = \color{black}, %% set frame color to avoid being affected by text color
basicstyle = \small \ttfamily , %% set listing font and size
breaklines = true, %% enable line breaking
numberstyle = \tiny,
}

% complex-systems.sty is the macro package for Complex Systems.
% It is available at
% http://www.complex-systems.com/samples/complex-systems.sty
% epsf.sty is the preferred graphics import method

\begin{document}

\title{Secure Cloud Computing:\\ {\large ORAM and Homomorphic Encryption}% 
% Use \\ to indicate line breaks in titles longer than about 
% 55 characters. 
%
}

\author{\authname{Riccardo Gennaro}\\[2pt] 
% Use \\[2pt] to end the line and add space between author name and affiliation. 
\authadd{Student ID: 3534219}\\
\authadd{Group 5}\\
\and
\authname{Péter Svelecz}\\[2pt] 
% Use \\[2pt] to end the line and add space between author name and affiliation. 
\authadd{Student ID: 3542629}\\
\authadd{Group 5}\\
}

% The following specifies the running headings 
%
% Each running heading should be less than about 50 characters long. 
% If necessary, give a shortened version of the title. 
%
% Use initials for first and second names. If all author names do not fit, truncate the 
% list and end with ``et al.''.
\markboth{Secure Cloud Computing 2024 Assignment 3} 
{ORAM and Homomorphic Encryption} 

\maketitle
% End title section

%\begin{abstract}
%    \section{Homomorphic Encryption over the Integers}

\subsection{Encrypting a bit vector}

We encrypted the bit vector provided in the assignment and added the resulting ciphertexts to the \texttt{swhe-task1.json} file. The ensure the correct way of encryption we made sure to use the quotient and modulo functions as defined in the scheme and described in the assignment text. Below you can find the resulting ciphertexts, each belonging to a bit message.

\subsection{Number of supported operations}

We tested the possible number of operations (homomorphic XOR and AND) for each ciphertext, using the provided parameters for the scheme. The reason behind the significant difference between the number of supported XOR and AND operations lies in the way these operations are implemented. Since XOR is implemented as integer addition, the generated noise during this operation accumulates way more slowly than iin the case of the AND operation which is implemented via integer multiplication (in both cases a $mod x\textsubscript{0}$ operation is also performed to decrease the generated noise). Because of this we were expecting similar results but it was interesting to see the actual numbers, as the difference is multiple orders of magnitude.

Following you can find the results for the number of supported operations.

\begin{verbatim}
Ciphertext 1 (Noise Bitlength: 1) supports:
  XOR operations: 40701
  AND operations: 5
Ciphertext 2 (Noise Bitlength: 3) supports:
  XOR operations: 105669
  AND operations: 3
Ciphertext 3 (Noise Bitlength: 5) supports:
  XOR operations: 12728
  AND operations: 1
Ciphertext 4 (Noise Bitlength: 7) supports:
  XOR operations: 62678
  AND operations: 2
Ciphertext 5 (Noise Bitlength: 9) supports:
  XOR operations: 75964
  AND operations: 1
\end{verbatim}

%\end{abstract}

%\begin{keywords}
%    Personal Health Record, PHR, Access Control, dCP-ABE
%\end{keywords}

\section{Homomorphic Encryption over the Integers}

\subsection{Encrypting a bit vector}

We encrypted the bit vector provided in the assignment and added the resulting ciphertexts to the \texttt{swhe-task1.json} file. The ensure the correct way of encryption we made sure to use the quotient and modulo functions as defined in the scheme and described in the assignment text. Below you can find the resulting ciphertexts, each belonging to a bit message.

\subsection{Number of supported operations}

We tested the possible number of operations (homomorphic XOR and AND) for each ciphertext, using the provided parameters for the scheme. The reason behind the significant difference between the number of supported XOR and AND operations lies in the way these operations are implemented. Since XOR is implemented as integer addition, the generated noise during this operation accumulates way more slowly than iin the case of the AND operation which is implemented via integer multiplication (in both cases a $mod x\textsubscript{0}$ operation is also performed to decrease the generated noise). Because of this we were expecting similar results but it was interesting to see the actual numbers, as the difference is multiple orders of magnitude.

Following you can find the results for the number of supported operations.

\begin{verbatim}
Ciphertext 1 (Noise Bitlength: 1) supports:
  XOR operations: 40701
  AND operations: 5
Ciphertext 2 (Noise Bitlength: 3) supports:
  XOR operations: 105669
  AND operations: 3
Ciphertext 3 (Noise Bitlength: 5) supports:
  XOR operations: 12728
  AND operations: 1
Ciphertext 4 (Noise Bitlength: 7) supports:
  XOR operations: 62678
  AND operations: 2
Ciphertext 5 (Noise Bitlength: 9) supports:
  XOR operations: 75964
  AND operations: 1
\end{verbatim}

\newpage
\section{Homomorphic Encryption over the Integers}

\subsection{Encrypting a bit vector}

We encrypted the bit vector provided in the assignment and added the resulting ciphertexts to the \texttt{swhe-task1.json} file. The ensure the correct way of encryption we made sure to use the quotient and modulo functions as defined in the scheme and described in the assignment text. Below you can find the resulting ciphertexts, each belonging to a bit message.

\subsection{Number of supported operations}

We tested the possible number of operations (homomorphic XOR and AND) for each ciphertext, using the provided parameters for the scheme. The reason behind the significant difference between the number of supported XOR and AND operations lies in the way these operations are implemented. Since XOR is implemented as integer addition, the generated noise during this operation accumulates way more slowly than iin the case of the AND operation which is implemented via integer multiplication (in both cases a $mod x\textsubscript{0}$ operation is also performed to decrease the generated noise). Because of this we were expecting similar results but it was interesting to see the actual numbers, as the difference is multiple orders of magnitude.

Following you can find the results for the number of supported operations.

\begin{verbatim}
Ciphertext 1 (Noise Bitlength: 1) supports:
  XOR operations: 40701
  AND operations: 5
Ciphertext 2 (Noise Bitlength: 3) supports:
  XOR operations: 105669
  AND operations: 3
Ciphertext 3 (Noise Bitlength: 5) supports:
  XOR operations: 12728
  AND operations: 1
Ciphertext 4 (Noise Bitlength: 7) supports:
  XOR operations: 62678
  AND operations: 2
Ciphertext 5 (Noise Bitlength: 9) supports:
  XOR operations: 75964
  AND operations: 1
\end{verbatim}



%%\printbibliography

\end{document}
